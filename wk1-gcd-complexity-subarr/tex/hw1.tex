\documentclass{article}
\usepackage{amsmath}
%\usepackage{algpseudocode}
\begin{document}

\textbf{Question 1a.}

Conditions:
\begin{itemize}
    \item $a,b,q,r$ are positive integers
    \item $0<a<b$
    \item $r<a$ and $0 \leq r < 1$
\end{itemize}

Rearrange
\begin{align*}
    \frac{a}{b} = \frac{1}{q + \frac{r}{a}}\\
    \frac{a}{b} = (q+{r}{a})^{-1} \\
    \frac{a}{b}(q+\frac{r}{a}) = (q+\frac{r}{a})^{-1}(q+\frac{r}{a}) \\
    \frac{a}{b}(q+\frac{r}{a}) = 1 \\
    \frac{qa}{b} + \frac{ra}{ab} = 1 \\
    \frac{qa}{b} + \frac{r}{b} = 1 \\
    \frac{qa+r}{b} = 1
\end{align*}

Solve for q
\begin{align*}
    \frac{qa+r}{b}=1 \\
    qa+r = b \\
    qa = b-r \\
    q = \frac{b-r}{a} \\
    q = \frac{b}{a} - \frac{r}{a}
\end{align*}
Since $r<a$ %and $0\leq r < 1$ 
and $a$ is an integer greater than $0$
this implies $\frac{r}{a} < r$. Of which, $r$ ranges from $(0,1)$. 
Implying $\frac{r}{a} \le r \le 1$.

Given, $q=\frac{b}{a} - \frac{r}{a}$ and that $q$ is an integer while 
$\frac{r}{a}$ is a fraction between $0$ and $1$. q 

Solve for r
\begin{align*}
    \frac{qa+r}{b} = 1 \\
    qa+r = b \\
    r = b-qa \\
    r = b -  a(\lfloor \frac{b}{a} \rfloor)
\end{align*}

\textbf{Question 2}

\begin{algorithm}
\end{algorithm}





\end{document}
